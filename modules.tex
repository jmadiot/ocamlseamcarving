\documentclass[a4paper]{article}
\usepackage[utf8]{inputenc}
\usepackage[T1]{fontenc}
\usepackage[normalem]{ulem}
\usepackage[french]{babel}
\usepackage{verbatim}
\usepackage{graphicx}

\begin{document}

\section{Module ppm}

Le module ppm gère l'ouverture des fichiers images. La fonction file 
prend le nom d'un fichier image en argument, convertit l'image à l'aide
d'ImageMagick (convert) et ouvre octet par octet le fichier.


\section {Module interface}

Ce module regroupe les fonctions de l'interface laborieuse. La modularité
accélère le développement des interfaces car l'objectif est fixé, et la 
manière dont l'utilisateur entre les données n'intervient pas, permettant 
ainsi au programmeur d'interface d'effectuer des modifications sans 
interférer avec les programmeurs utilisant le module. (C'est d'autant plus
vrai pour les interfaces que le développement d'une nouvelle fonctionalité
implique souvent de réfléchir à une interface correspondante.

\section {Module SeamCarving}

Ce module regroupe les fonctions que l'utilisateur final utilisera :
(description des fonctions)

Il



\end{document}
